Dear Raffi, 

We regret to inform you that your workshop proposal entitled 

  FOOPL: International Workshop on Functional-Inspired Language Features in Mainstream Object-Oriented Programming Languages 

has not been accepted. The program committee selected 31 out of 44 submissions, with 10 workshops being merged together. All submissions were reviewed by, at least, three members of the program committee and discussed online. Due to the limited space, the competition is tight, and some good workshop proposals unfortunately could not be accepted. 

Please find at the end of this email your proposal�s reviews. We hope that they will help you improve your proposal for a future submission. 

We would like to thank you again for your interest in organizing a workshop at ICSE 2019, and hope that we can still see you in Montreal! 

Best regards, 

Sven Apel and David Lo
ICSE 2018 Workshop Co-Chairs

https://conf.researchr.org/home/icse-2019


----------------------- REVIEW 1 ---------------------
PAPER: 978
TITLE: FOOPL: International Workshop on Functional-Inspired Language Features in Mainstream Object-Oriented Programming Languages
AUTHORS: Raffi Khatchadourian and Mehdi Bagherzadeh


----------- Overall Evaluation -----------
FOOPL aims to promote the integration of functional language features into object-oriented programming. There are numerous recent examples of such new language features in mainstream programming languages. FOOPL would be the first version of the workshop.

+ emerging topic

- topic too narrow
- expertise of co-chairs
- diversity of PC

The examples given in the proposal are evidence that the workshop focuses on an emerging and relevant topic. However, the topic seems to be too narrow for such a broad conference as ICSE. It is questionable whether there will be enough submissions and attendances. A conference focusing on programming languages might be a better fit. There are also doubts that the expertise and diversity of expertise by the co-chairs and the PC is sufficient. In particular, there are PC members from the same institution.


----------------------- REVIEW 2 ---------------------
PAPER: 978
TITLE: FOOPL: International Workshop on Functional-Inspired Language Features in Mainstream Object-Oriented Programming Languages
AUTHORS: Raffi Khatchadourian and Mehdi Bagherzadeh


----------- Overall Evaluation -----------
Summary: This proposal begins with a discussion of challenges in
programming using functional features in OO languages. Then, a one-day
workshop is proposed for programmers and researchers to discuss these
issues, and evolve solutions. The components of the workshop would be a
keynote address, presentations of reviewed papers, and a breakout session. 

Pluses:

1 This is an important area for scientific investigation and progress. 

2 Proposed list of PC members is mentioned. 

Minuses:

1 The topic of this workshop is very specialized, and focused around
  PL. Hence, it may not be a good fit for a broad-ranging and SE focused
  conference like ICSE.

2 The first page of the proposal gives a lot of examples regarding
  challenges in programming with functional features.  However, the rest of
  the proposal does not seem to connect well with this discussion.

2 There is little discussion around what connections the proposers have
  with the programmer or researcher community that is interested in this
  area. Hence, it is not clear how they will ensure good participation and
  impact.

3 A lot of importance is given to the keynote speaker. However, no
  candidate is identified.


----------------------- REVIEW 3 ---------------------
PAPER: 978
TITLE: FOOPL: International Workshop on Functional-Inspired Language Features in Mainstream Object-Oriented Programming Languages
AUTHORS: Raffi Khatchadourian and Mehdi Bagherzadeh


----------- Overall Evaluation -----------
The theme of this workshop is to discuss language features for object-oriented languages. Personally, I think such discussion interesting and helpful for the design of object-oriented languages. 

However, I don't think the audience of this workshop have much overlap with the audience of ICSE. This makes this workshop proposal not very suitable for ICSE'19. A more suitable conference might be OOPSLA. 

This workshop further restricts the topics into the scope of functional-inspired features making the audience of this workshop smaller. First, I don't think many participants of ICSE'19 have enough experience with functional programming languages. Second, except for the MapReduce example, it seems that there are not many interesting issues in functional-inspired language features in object-oriented languages. 

This is the first edition of this workshop series. There has not been evidence that such a workshop can attract enough submissions from the SE community.

