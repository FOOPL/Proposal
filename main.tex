\documentclass[10pt, conference]{IEEEtran}
% \documentclass[sigconf]{acmart}

% ACM stuff
% \setcopyright{rightsretained}
% \acmDOI{10.475/123_4}

% ISBN
% \acmISBN{123-4567-24-567/08/06}

%Conference
% \acmConference[ICSE'18]{International Conference on Software Engineering}{May 2018}{Gothenburg, Sweden} 
% \acmYear{2017}
% \copyrightyear{2017}
% \acmArticle{4}
% \acmPrice{15.00}
% ACM stuff (end)

\usepackage{color}
\usepackage{url}
%\usepackage{cite}
\newcommand{\todo}[1]{{\textcolor{red}{\textbf{TODO:}~#1}}} 
\newcommand{\sa}[1]{{\textcolor{red}{\textbf{Sven:}~#1}}}  
\newcommand{\sn}[1]{{\textcolor{green}{\textbf{Sarah:}~#1}}}  
\newcommand{\rk}[1]{{\textcolor{red}{\textbf{Raffi:}~#1}}} 
%\newcommand{\updated}[1]{{\textcolor{blue}{#1}}}
\newcommand{\updated}[1]{#1}
\newcommand{\shortname}{FOOPL}

\begin{document}

\title{\shortname{}: International Workshop on Functional-Inspired Language Features in Mainstream Object-Oriented Programming Languages}

% author names and affiliations
% use a multiple column layout for up to two different
% affiliations
% \newcommand\tud[0]{\textsuperscript{\normalfont \textdagger}}
% \newcommand\iowa[0]{\textsuperscript{\normalfont \textparagraph}}
% \newcommand\utd[0]{\textsuperscript{\normalfont \ddag}}
% \newcommand\lanu[0]{\textsuperscript{\normalfont \textsection}}
% \newcommand\ualberta[0]{\textsuperscript{\normalfont \textasteriskcentered}}

% authors
% IEEE authors
%\author{\IEEEauthorblockN{Sven Amann\textsuperscript{\textdagger}, Sarah Nadi\textsuperscript{\ddag}, Hoan Anh Nguyen\textsuperscript{\textsection}, Tien N. Nguyen\textsuperscript{*}, Raffi Katchadourian\textsuperscript{$\top$}}
%\IEEEauthorblockA{Technische Universit\"{a}t Darmstadt\textsuperscript{\textdagger}, University of Alberta\textsuperscript{\ddag}, Iowa State University\textsuperscript{\textsection},\\ University of Texas at Dallas\textsuperscript{*}, City University of New York\textsuperscript{$\top$}}
%}
\author{\IEEEauthorblockN{Raffi Khatchadourian}
\IEEEauthorblockA{\textit{City University of New York (CUNY) Hunter College}\\
raffi.khatchadourian@hunter.cuny.edu}
\and
\IEEEauthorblockN{Mehdi Bagherzadeh}
\IEEEauthorblockA{\textit{Oakland University}\\
mbagherzadeh@oakland.edu}
}
% ACM authors
% \author{Sven Amann}
% \affiliation{Technische Universit\"{a}t Darmstadt}
% \author{Sarah Nadi}
% \affiliation{University of Alberta}
% \author{Hoan Anh Nguyen}
% \affiliation{Iowa State University}
% \author{Tien N. Nguyen}
% \affiliation{University of Texas at Dallas}
% \author{Raffi Katchadourian}
% \affiliation{City University of New York}
% authors (end)

% ACM stuff

%\begin{CCSXML}
%\end{CCSXML}

%\ccsdesc[500]{Computer systems organization~Embedded systems}
%\ccsdesc[300]{Computer systems organization~Redundancy}
%\ccsdesc{Computer systems organization~Robotics}
%\ccsdesc[100]{Networks~Network reliability}

%\keywords{ACM proceedings, \LaTeX, text tagging}

% ACM stuff (end)

% IEEE wants it here
\maketitle

\begin{abstract}
	\input{abstract.txt}
\end{abstract}

% ACM wants it here
% \maketitle

\begin{IEEEkeywords}
	functional programming, object-oriented programming, fusion programming paradigms, APIs, streaming
\end{IEEEkeywords}

\section{Organizer Contact Information}

\begin{itemize}
	\item \textbf{Raffi Khatchadourian (main contact),} City University of New York (CUNY) Hunter College, Email: raffi.khatchadourian@hunter.cuny.edu
	\item \textbf{Mehdi Bagherzadeh,} Oakland University, Email: mbagherzadeh@oakland.edu
\end{itemize}

\section{Motivation and Objectives}
%Motivation of the workshop's relevance to the field of software engineering.

The use of constructs from functional programming languages in mainstream Object-Oriented languages is becoming increasingly pervasive~\cite{Biboudis2015}. For example, developers can now use lambdas, stream, and MapReduce-style~\cite{Dean2008} computations in languages such as Java~\cite{OracleCorporation2017} (and Android~\cite{Lau2017}), C\#~\cite{Microsoft2018}, F\#~\cite{fsharp} and Scala~\cite{Lausanne2015}, to name a few. Through these mechanisms, developers can declaratively process native data structure like collections and infinite data structures. Streaming Application Programmer Interfaces (APIs), for instance, available in many mainstream Object-Oriented (OO) programming languages, support this paradigm fusing. Other mechanisms include lambdas, i.e., (typically stateless) first-class computational units facilitating deferred execution~~\cite{OracleCorporation2015,Wagner2017}, and option types, i.e., nomadic types supporting MapReduce-style operations.

These language features have many benefits~\cite[Ch.~1]{Warburton2014}, including succinct~\cite{Mazinanian2017} and nearly effortless (syntax-wise) parallelism~\cite{OracleCorporation2015a}. However, combining the paradigms may result in code that behaves in ways not expected by developers. For instance, unexpected errors and suboptimal performance may be incurred as subtle considerations are required to produce code that is correct, optimally parallelizable, efficient, reliable, and free of bugs related to nontermination, nondeterminism, and mismanaged resource cleanup in using these constructs~\cite{Oracle2017d}. Moreover, constructs and APIs may not behave in ways developers expect. Developers may also not prefer to use these constructs to write concurrent code~\cite{Nielebock2018}, perhaps missing opportunities where this modern technology may be beneficial~\cite{Tang2018}. And, API developers may be restricted in writing libraries that are maximally usable and extendable due to current language design~\cite{Biboudis2015}.

One significant problem is that MapReduce is traditionally highly distributed with no shared memory, whereas streaming APIs in mainstream OO languages execute within a single node under multiple threads sharing the same memory. This difference makes using these constructs and APIs prone to such problems as thread contention, necessitating great care in writing correct yet efficient code. Moreover, discovering and repairing these problems may involve complex interprocedural whole-program analysis, a thorough understanding of construct intricacies, and detailed knowledge of situational API replacements. Manually detecting problematic code areas (smells~~\cite{Tufano2015,Fowler2008}) and uncovering appropriate optimizations that transform source code while preserving its original semantics (refactoring~\cite{Opdyke1992,Fowler1999}) can be daunting~\cite{Dig2009}, especially in large and complex projects or where developers are unfamiliar with how best to use such constructs and APIs.

\subsection{Goals and Outcomes}\label{sec:goals}
%Anticipated goals and outcomes of the workshop (e.g., open research problems to pursue, validation objectives, empirical studies).

The 1\textsuperscript{st} International Workshop on Functional-inspired Language Features in Mainstream Object-Oriented Programming Languages (FOOPL) aims to bring together researchers, practitioners, and educators interested in this emerging topic with the goal of sharing results and ideas in solving crucial problems in the mass adaptation, correct usage, education, language and API design and implementation, and tool support of functional-inspired language constructs in mainstream OO languages. The workshop will also cover a broader view of this technology by incorporating additional artifacts beyond code, including requirements, specification, and tests for this new paradigm.

Researchers, practitioners, and educators will be encouraged to engage in thought-provoking discussions surrounding these topics so that this relatively new technology has the best support for success. The workshop will also provide a productive and constructive setting for industry and academia collaboration. Our goal is to foster new and existing collaborations, help elevate working ideas into mature conference or journal papers or industrial ideas, and discover best practices in incorporating the use of these constructs and APIs in (possibly) beginning Computer Science education.

\section{Format and Required Services}

%Workshop format (e.g., paper presentations, keynotes, breakout sessions, panel-like discussions) and plans for generating and stimulating discussions.
%Desired length of the workshop (i.e. 1 or 2 days).
%Special services that are needed; logistic and/or equipment constraints.

In our first edition of this workshop, we plan to have three sessions with presentations and one session with a discussion at the end. We would like to bring up many interesting challenges and open work during the discussions, as well as help ongoing projects flourish. To help identify open problems, in preparation of the workshop, we will derive a set of questions from the accepted submissions and ask participants for feedback, to ensure that the questions adequately reflect the current challenges that the community faces. These questions will serve as a framework for the workshop discussions and to keep them focused on actionable ideas. 

We propose a 1-day workshop with the following format:

\begin{itemize}

	\item \textbf{Morning keynote speaker.} This will either be an established researcher who provides some background of the area and the open challenges, practitioner who can highlight the current challenges faced in industry, or an educator interested in teaching students the best way to incorporate functional-inspired constructs and API into curriculum utilizing mainstream Object-Oriented programming.

	\item \textbf{Presentation session.} The presentations will be kept short and focus on presenting the main open challenges of the respective work, as inspiration for the following breakout sessions.

	\item \textbf{Breakout sessions.} We will have breakout sessions in the afternoon, where participants will be divided into different groups. We will encourage the mixing of people from academia and industry. The goal of these sessions is to enable more in-depth discussion of challenges presented in the talks. This will also help those interested in similar topics to foster new collaborations. We will provide guidelines of what the outcomes of these discussions will be, such that participants can present their results at the end of the workshop.

\end{itemize}

To run the workshop, we would need a room that is equipped with a projector and a flip chart/whiteboard to record the ideas generated by the attendees.

\section{Target Audience}

Our goal is to have 20--40 participants from both industry and academia. Given that this will be the first year of the workshop and that this area is relatively new, we expect around 20 participants.

\subsection{Attendee Background} 
%Required background of the workshop attendees.
%Whether a Mix of industry and research participants is being sought.
%Desired minimum and maximum number of workshop participants, expected number of participants.

We will encourage a diverse set of attendees to foster interesting discussion at the workshop. Backgrounds may include program analysis, software evolution, formal methods (for specification), empirical software engineering, requirements engineering, software testing and transformation, run time verification, language and tool engineering, and search-based software engineering. We believe that this wide variety of background will naturally lend itself to attract people from both academia and industry.

\subsection{Participant Selection and Solicitation}
%Participant solicitation and selection process, including whether the workshop will be open or closed.

The workshop will be open to anyone who would like to attend and contribute to the discussion. While papers will be peer-reviewed, everyone is welcome to register for attending the workshop. Participants will be solicited by various means, including prominent email lists and social media. We will also identify industry \emph{and} academic conferences for which fliers will be placed. Since this is the first year of the workshop, we will also ask our program committee members to help spread the word about the workshop.

\section{Proceedings}
%How many and what type of contributions will be solicited (number of pages and type: extended abstracts, position and/or research papers, etc.).

We will solicit research papers, practice papers, position papers, and experience reports. To encourage more discussion, papers will be limited to four pages. To attract local industry participation, we will also call for talk abstracts. Such talks can focus on challenges faced in industry and motivate future research projects. We will seek assistance from the local ICSE organizers to contact local practitioners and educators and forward the workshop's CfP to relevant industry and academic participants.

\subsection{Evaluation Process}
%Evaluation process that will be used to decide the acceptance of submissions.
%A list of program committee members, including proposed and committed members.

All four types of submissions will be peer-reviewed by three program committee (PC) members. The submissions will be evaluated on relevance to the workshop topic, interest level of the anticipated participants, presentation and organization of ideas, and potential for engaging discussion at the workshop. Below is a list of the proposed committee members, where names in bold those who have already agreed to serve:

\begin{enumerate}

	\item \textbf{Aggelos Biboudis, École Polytechnique Fédérale de Lausanne}

	\item \textbf{Yoonsik Cheon, University of Texas at El Paso}

	\item \textbf{Shigeru Chiba, University of Tokyo}

	\item \textbf{David Chiu, University of Puget Sound}

	\item Julian Dolby, IBM T. J. Watson Research Center

	\item \textbf{Robert Dyer, Bowling Green State University}

	\item \textbf{George Fourtounis, University of Athens}

	\item Lyle Franklin, Pivotal, Inc.

	\item \textbf{Alex Gyori, Facebook}

	\item Jan Lahoda, Oracle Corporation

	\item \textbf{Yuheng Long, Google}

	\item \textbf{Hidehiko Masuhara, Tokyo Institute of Technology}

	\item \textbf{Davood Mazinanian, University of British Columbia}

	\item \textbf{Hua Ming, Oakland University}

	\item \textbf{Sebastian Nielebock, Otto von Guericke University}

	\item \textbf{Hoan Nguyen, Iowa State University}

	\item Frank Ortmeier, Otto von Guericke University

	\item \textbf{Subash Shankar, City University of New York (CUNY) Hunter College}

	\item \textbf{Nikolaos Tsantalis, Concordia University}

	\item \textbf{Ganesha Upadhyaya, Huawei R\&D}

	\item Titus Winters, Google

	\item Erik Wittern, IBM T. J. Watson Research Center

	\item Neng-Fa Zhao, City University of New York (CUNY) Brooklyn College

\end{enumerate}

\subsection{Call for Papers}
%Links to preliminary web site of the workshop and call for papers.

The preliminary website for the workshop, as well as the call for papers, can be found here:

\begin{center}
  \texttt{\url{http://foopl.github.io}}
\end{center}

\section{Workshop History}

This will be the first year of the workshop.

\section{Organizers' Bios}
%a brief description of each organizer's background, including relevant past experience in organizing conferences and workshops.

\begin{itemize}

	\item \textbf{Dr.~Raffi Khatchadourian} is an assistant professor City University of New York (CUNY) Hunter College. His research interests include program analysis, program transformation, and automated refactoring, including its impact on API consumption. He helped organize the LaMod'16 workshop at MODULARITY'16 and WAPI'17 at ICSE'18. He has also served as the web chair for ECOOP'11 and is currently serving as the publicity chair for ESEC/FSE'18.

	\item \textbf{Dr.~Mehdi Bagherzadeh} is an assistant professor at Oakland University. His research interest is in making engineering of correct concurrent software easier, working at the intersection of programming languages, formal methods and software engineering. He has been on the organizing committee of ESE/FSE'18, PLRP@COMPSAC'18, and the Midwest Big Data Summer School '16.

\end{itemize}

\section*{Acknowledgment}

We would like to thank Sebastian Nielebock and Robert Dyer for their insightful comments and helpful feedback regarding this proposal. Portions of this proposal are based on the WAPI'18 proposal of ICSE'18.

% IEEE references
\bibliographystyle{IEEEtranS}
% ACM references
% \bibliographystyle{ACM-Reference-Format}
\bibliography{refs}

\newpage
\onecolumn

\begin{center}
	\textbf{Call For Papers}
\end{center}

Using functional programming language constructs in mainstream Object-Oriented languages is becoming increasingly pervasive. For example, developers can now use lambdas, stream, and MapReduce-style computations in languages such as Java, C\#, F\# and Scala, to name a few. Through these mechanisms, developers can declaratively process native data structure 
% like collections
and infinite data structures. Streaming Application Programmer Interfaces (APIs), for instance,
% available in many mainstream Object-Oriented (OO) programming languages, 
support this paradigm fusing. Other mechanisms include lambdas
% , i.e., (typically stateless) first-class computational units facilitating deferred execution, 
and option types, i.e., nomadic types supporting MapReduce-style operations.

These language features have many benefits, including succinct and nearly effortless
% (syntax-wise)
parallelism. However, combining the paradigms may result in code that behaves in ways not expected by developers. For instance, unexpected errors and suboptimal performance may be incurred as subtle considerations are required to produce code that is correct, optimally parallelizable, efficient, reliable, and free of bugs related to nontermination, nondeterminism, and mismanaged resource cleanup in using these constructs. Developers may also not prefer to use these constructs to write concurrent code, perhaps missing opportunities where this modern technology may be beneficial.
% And, API developers may be restricted in writing libraries that are maximally usable and extendable due to current language design.

One significant problem is that MapReduce is traditionally highly distributed with no shared memory, whereas streaming APIs in mainstream OO languages execute within a single node under multiple threads sharing the same memory. This difference makes using these constructs and APIs prone to 
% such problems as 
thread contention, necessitating great care in writing correct yet efficient code. Moreover, discovering and repairing these problems may involve complex interprocedural whole-program analysis, a thorough understanding of construct intricacies, and detailed knowledge of situational API replacements. Manually detecting problematic code areas (smells) and uncovering appropriate optimizations that transform source code while preserving its original semantics (refactoring) can be daunting, especially
% in large and complex projects or 
when developers are unfamiliar with how best to use such constructs and APIs.

The 1st International Workshop on Functional-inspired Language Features in Mainstream Object-Oriented Programming Languages (FOOPL) aims to bring together researchers, practitioners, and educators interested in this emerging topic with the goal of sharing results and ideas in solving crucial problems in the mass adaptation, correct usage, education, language and API design and implementation, and tool support of functional-inspired language constructs in mainstream OO languages. The workshop will also cover a broader view of this technology by incorporating additional artifacts beyond code, including requirements, specification, and tests for this new paradigm. Researchers, practitioners, and educators will be encouraged to engage in
% thought-provoking 
discussions surrounding these topics so that this relatively new technology has the best support for success. The workshop will also provide a productive and constructive setting for cross-boundary collaboration.

% \noindent
% \textbf{Topics}

Topics of interest include, but are not limited to:

\begin{itemize}

	\item Design and implementation of functional-inspired language and streaming APIs in mainstream Object-Oriented languages.

	\item Extensibility of streaming APIs and functional-inspired language features and constructs.

	\item Empirical results on how developers (expect to) use current streaming APIs and lambdas in mainstream OO languages.

	\item Assessing upcoming language features and APIs.

	\item Issues arising from the interplay of functional and mainstream OO paradigms.

	\item Refactoring, migrating, and re-engineering of mainstream OO systems to make use of functional-inspired language features.

	\item Using functional
		% language
		constructs in other software engineering artifacts
		% and activities 
		such as requirements, documentation, and specification.

	\item Bug and suboptimal performance detection in the (mis)use of streaming APIs and lambdas.

	\item Quality metrics for assessing 
		% functional-inspired 
		language features
		% or streaming API 
		usability in mainstream OO software 
		% API and language 
		(engineer perspective).

	\item Quality metrics for assessing the actual usage of 
		% functional-inspired 
		language features
		% or streaming APIs 
		% (API and language consumer perspective).
		(consumer perspective).

	\item Automated repair of programs that use streaming APIs, lambdas, or other function-inspired constructs or APIs.

	\item Automated testing of mainstream OO programs using functional-inspired constructs.

	\item Using streaming APIs and lambdas in Computer Science education.

	\item Concurrency using streaming APIs and lambdas.

	\item Static analyses for programs using functional-inspired features in OO languages.

	\item Run-time support for improving the efficiency of streaming APIs.

	\item Identification of open challenges and proposed solutions.

	\item Synergies between
		% streaming API/
		functional-inspired mainstream OO programming language research
		% challenges 
		and other research areas.

\end{itemize}

\noindent
\textbf{\large Submission Information}

\shortname{} 2018 invites contributions in the form of papers that are no more than \textbf{4 pages} in length. Submissions can stem from either academia or industry. We also welcome industrial and educational talk abstracts. Research papers, practice papers, position papers, and experience reports are all welcomed. All submissions should describe unpublished work and must have been neither previously accepted for publication nor concurrently submitted for review in another journal, book, conference, or workshop. Submissions are peer-reviewed and accepted papers will appear in the workshop proceedings.

\noindent
\textbf{\large Important Dates}

\textbf{Submissions due}: February 1, 2019; \textbf{Notification:} March 1, 2019; \textbf{Camera-ready copies due:} March 15, 2019

% The official publication date of the workshop proceedings is the date the proceedings are made available in the ACM Library. This date may be up to two weeks prior to the first day of ICSE 2019. The official publication date affects the deadline for any patent filings related to published work.

\end{document}
